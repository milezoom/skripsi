\documentclass{article}
\usepackage[T1]{fontenc}
\usepackage[utf8]{inputenc}
\usepackage{lmodern}
\usepackage[a4paper,left=4cm,top=3cm,right=3cm,bottom=3cm]{geometry}
\usepackage{graphicx}
\usepackage{ragged2e}
\usepackage{tabulary}
\usepackage[english]{babel}
\usepackage{blindtext}
\usepackage{fancyhdr}

\makeatletter
\newcommand\tabfill[1]{%
\dimen@\linewidth%
\advance\dimen@\@totalleftmargin%
\advance\dimen@-\dimen\@curtab%
\parbox[t]\dimen@{\raggedright #1\ifhmode\strut\fi}%
}

\begin{document}


\begin{titlepage}
\begin{figure}
  \centering
  \includegraphics{logo_UI_hitam.png}
\end{figure}
\begin{center}
  \textbf{UNIVERSITAS INDONESIA}\\
  \vfill
  \textbf{KORELASI PENYEBARAN INFORMASI MELALUI MEDIA SOSIAL DENGAN TINGKAT POPULARITAS SUATU TEMPAT WISATA}\\
  \vfill
  \textbf{PROPOSAL PENELITIAN}\\
  \vfill
  \textbf{MUSLIM}\\
  \textbf{1206208845}\\
  \vfill
  \textbf{FAKULTAS ILMU KOMPUTER}\\
  \textbf{PROGRAM STUDI SISTEM INFORMASI}\\
  \textbf{DEPOK}\\
  \textbf{MEI 2016}
\end{center}
  \thispagestyle{empty}
\end{titlepage}


\renewcommand{\abstractname}{\textbf{ABSTRAK}}
\begin{abstract}
    \vspace{10pt}
    \begin{tabbing}
        \hspace*{2.5cm}\=\hspace*{0.2cm}\=\hspace*{10cm}\kill
        Nama \> : \> Muslim\\
        Program Studi \> : \> Sistem Informasi\\
        Judul \> : \> \tabfill{
        Aplikasi Pencari Pekerjaan Dengan Memanfaatkan Perangkat Internet Of Things 
        Dan Konsep Activity Feed}
    \end{tabbing}
    \vspace{10pt}
    \blindtext
\end{abstract}
\newpage


\section{Pendahuluan}
\subsection{Latar Belakang}
    Media sosial seringkali digunakan untuk menyebarkan informasi.
    Salah satu informasi yang disebarkan adalah tempat wisata.
    Adanya informasi ini membuat sebuah tempat wisata menjadi lebih mudah dijangkau.
\subsection{Rumusan Masalah}
    
\subsection{Pertanyaan Penelitian}
\subsection{Tujuan Penelitian}
\subsection{Manfaat Penelitian}
\subsection{Ruang Lingkup Penelitian}
\newpage


\section{Studi Literatur}
\subsection{Sub1}
\subsection{Sub2}
\subsection{Sub3}
\newpage


\section{Metodologi}
\subsection{Analisis Masalah}
\subsection{Studi Literatur}
\subsection{Perumusan Hipotesis}
\subsection{Perumusan Instrumen Penelitian}
\subsection{Pengumpulan Data}
\subsection{Analisis Data}
\subsection{Penarikan Kesimpulan}
\newpage


\section{Expektasi Hasil}

\end{document}
