\documentclass[]{article}
\usepackage[a4paper,top=3cm,right=3cm,bottom=3cm,left=4cm]{geometry}
\usepackage[T1]{fontenc}
\usepackage[utf8]{inputenc}
\usepackage{lmodern}
\usepackage{graphicx}
\usepackage{setspace}
\usepackage{blindtext}
\usepackage{indentfirst}
\usepackage{hyperref}
\usepackage{url}
\usepackage[indonesian]{babel}
\usepackage{csquotes}
\usepackage[autolang=hyphen,backend=biber,style=authoryear,citestyle=apa,sorting=ynt]{biblatex}
\usepackage{fancyhdr}

\pagestyle{fancy}
\fancyhf{}
\rhead{\thepage}
\rfoot{\textsf{\textbf{Universitas Indonesia}}}
\renewcommand{\headrulewidth}{0pt}

\addbibresource{references.bib}
\urlstyle{same}
\renewcommand*{\finalnamedelim}{%
  \ifnumgreater{\value{liststop}}{2}{\finalandcomma}{}%
  \addspace dan\space}


\begin{document}
\begin{spacing}{1.5}

\pagenumbering{roman}

\begin{titlepage}
\begin{center}    
\includegraphics{logo_UI_hitam.png}\\
\uppercase{\textbf{universitas indonesia}}
\vfill
\uppercase{\textbf{<judul menyusul>}}
\vfill
\uppercase{\textbf{skripsi}}
\vfill
\uppercase{\textbf{Muslim}}\\
\uppercase{\textbf{1206208845}}
\vfill
\uppercase{\textbf{fakultas ilmu komputer}}\\
\uppercase{\textbf{program studi sistem imformasi}}\\
\uppercase{\textbf{depok}}\\
\uppercase{\textbf{mei 2016}}\\
\end{center}
\end{titlepage}

\renewcommand{\abstractname}{\large Abstrak}
\begin{abstract}
\begin{center}
    \begin{tabular}{l l p{8.78cm}}
        Nama & : & Muslim \\
        Program Studi & : & Sistem Informasi \\
        Judul Skripsi & : & Lorem ipsum dolore sit amet consectuere adipising elit sit amet.
        Seq quidila punco ren terate men ano in quidispil ren cut. 
        Rom ela et ema dis tanpi ruk mis cer.
        Lerru ridista men quo linca nen yukore.
    \end{tabular}
\end{center}
\vspace{2em}
\blindtext
\vfill
\end{abstract}
\newpage

\renewcommand{\contentsname}{Daftar Isi}
\tableofcontents
\addtocontents{toc}{~\hfill\textbf{Halaman}\par}
\lhead{}
\newpage

\pagenumbering{arabic}

\section{Pendahuluan}
\subsection{Latar Belakang}
Pengguna media sosial di Indonesia cukup tinggi. Dari 255,5 juta penduduk yang ada di Indonesia,
terdapat 72,7 juta sambungan internet yang digunakan (\cite{SK2015}). Kemudian, terdapat 72 juta
akun media sosial aktif di Indonesia, dan 62 juta darinya diakses menggunakan sambungan 
\textit{mobile} (\cite{SK2015}). Dengan demikian, walaupun memang tidak cukup besar, total akun
media sosial (sekitar 24\% dari total penduduk di Indonesia) yang ada di Indonesia cukup banyak.

Komentar di media sosial dapat digunakan untuk menentukan preferensi seseorang terhadap suatu
produk. Analisis yang dilakukan pada komentar tersebut dapat memisahkan fitur-fitur apa saja yang
menjadi perhatian bagi konsumen produk tersebut. Dari penilaian terhadap fitur-fitur tersebut,
kemudian dapat ditentukan apakah produk tersebut disukai oleh konsumen (\cite{SM2012}).


\begin{itemize}
    \item Disisi lain procurement yang dilakukan perusahaan harus setepat mungkin agar tidak memberatkan finansial [buku]
    \item Procurement ini berkaitan dengan jumlah produk yang akan diproduksi [ide-sendiri]
    \item Planning procurement yang saat ini dilakukan kebanyakan bergantung dari hasi prediksi penjualan masa lalu [web]
    \item Proses planning ini dapat ditingkatkan akurasinya dengan bantuan data dari media sosial yang lebih cepat di
        update [ide-sendiri]
\end{itemize}
\par
\begin{itemize}
    \item 
\end{itemize}
\par
\subsection{Rumusan Masalah}
\newpage

\section{Tinjauan Pustaka}
\subsection{\textit{Sentiment Analysis} untuk menentukan Produk}
\subsection{Metode untuk \textit{Procurement Planning}}
\newpage

\renewcommand{\refname}{Daftar Pustaka}
\printbibliography

\end{spacing}
\end{document}
